\documentclass{mini}
\usepackage[utf8]{inputenc}
\usepackage[polish]{babel}
\usepackage{enumitem}
\usepackage{todonotes}
\presetkeys
	{todonotes}
	{inline}{}
\usepackage{listings}
\usepackage[lighttt]{lmodern}
\usepackage{xcolor}
\lstset{
    backgroundcolor=\color{yellow!20},%
    basicstyle=\small\ttfamily,%
    numbers=left, numberstyle=\color{gray}\tiny, stepnumber=1
}
\lstset{keywords={%  
		ACS,  after,  always, at, causes, ever,
		generally,  if, imposible,  initially,  
		invokes,  involved, -,  OBS,  perfomed,
		releases, triggers, typically,  when
    },keywordstyle={\bfseries},
    frame=single,
    breaklines=true,
    postbreak=\raisebox{0ex}[0ex][0ex]{\ensuremath{\color{red}\hookrightarrow\space}}
}  

%------------------------------------------------------------------------------%
\title{Scenariusze działań z efektami domyślnymi}
\pm{Tomasz Janiszewski}
\author{Tomasz Janiszewski
\\Jakub Dutkowski
\\Tomasz Łącki
\\Rafał Wyka
\\Maciej Bednarz}
\monthyear{\today}
%------------------------------------------------------------------------------%

\begin{document}
\maketitle
\tableofcontents

\newpage

\section{Opis zadania}
Zadaniem projektu jest opracowanie i zaimplementowanie:
\begin{itemize}
\item języka akcji pewnej klasy systemów dynamicznych,
\item języka kwerend, zapewniającego uzyskanie odpowiedzi na określone pytania.
\end{itemize}
Szczegółowy opis klasy systemów dynamicznych oraz języka akcji jest opisany w rozdziale 'Opis języka akcji',
natomiast język kwerend oraz~zadawane pytania znajdują się w rozdziale 'Opis języka kwerend'.
W tym dokumencie znajdują się również przykłady.
Pokazują one konkretne przypadki użycia oraz oczekiwane wyniki działania programu.

\section{Język akcji}

Poniższy język został zaprojektowany tak aby spełnić następujące warunki:
\begin{itemize}
	\item Prawo inercji
	\item Linowy model czasu (czas dyskretny)
	\item Skewencyjność działań
	\item Niedeterminizm
	\item Z każdą akcją związane są
		\begin{itemize}
			\item Waruneki początkowe
			\item Skutki
				\begin{itemize}
					\item[środowskowe] 
						\begin{itemize}
							\item[pewne] występuje zawsze po zakończeniu akcji
							\item[typowe] występują zazwyczaj po zakończeniu akcji
						\end{itemize}
					\item[dynamiczne] występują po czasie $d \geqslant 0$
						\begin{itemize}
							\item[pewne] występuje zawsze po zakończeniu akcji
							\item[typowe] występują zazwyczaj po zakończeniu akcji
						\end{itemize}
				\end{itemize}
			\item Wykonawca (agent)
			\item Czas trwania $t = 1$
		\end{itemize}	
	\item Pewne stany mogą rozpoczynać wykonywanie pewnych akcji
	\item Agenci mogą nie potrafić wykonać akcji w pewnych stanach. Stany te są określone przez podanie
	warunków lub konkretnych punktów czasowych.	
\end{itemize}
Językiem odpowiadającym powyższym warunkom jest jezyk $\mathcal{AL}$ 
opisujący domeny akcji z czasem liniowym.

\subsection{Opis symboli języka}
\begin{definition}
Język definujemy jako trójkę
\begin{equation}
\psi = \left( \mathcal{F}, \mathcal{A}, \mathcal{T} \right)
\end{equation}
\begin{itemize}
\item[$\mathcal{F}$] niepusty zbiór inercji (dalej \emph{fluenty})
\item[$\mathcal{A}$] niepusty zbiór akcji (dalej \emph{akcje})
\item[$\mathcal{T}$] niepusty zbiór punktów czasu, $\mathcal{T} = \mathbb{N}$
\end{itemize}
\end{definition}

\subsection{Syntaktyka języka}
\todo[inline]{nie jestem pewien czy tak to powinno wyglądać}
\begin{description}[style=nextline]
	\item[$\texttt{initially } \alpha$]	
	stan początkowy fluentów w formule $\alpha$
	\item[$a_i \texttt{ causes } \alpha \texttt{ if } \pi$]
	akcja $a_i$ powoduje przejście w stan $\alpha$ jeśli zachodzi warunek $\pi$
	\item[$\texttt{typically } a_i \texttt{ causes } \alpha \texttt{ if } \pi$]
	zazwyczaj akcja $a_i$ powoduje przejście w stan $\alpha$ jeśli zachodzi warunek $\pi$
	\item[$a_i \texttt{ invokes } a_j \texttt{ after } d \texttt{ if } \pi$] 
	akcja $a_i$ powoduje wykonanie akcji $a_j$ po $d$ chwilach od zakończenia akcji $a_i$,
	jeśli zachodzi warunek $\pi$
	\item[$\texttt{typically } a_i \texttt{ invokes } a_j \texttt{ after } d \texttt{ if } \pi$] 
	zazwyczaj akcja $a_i$ powoduje wykonanie akcji $a_j$ po $d$ chwilach od zakończenia akcji $a_i$,
	jeśli zachodzi warunek $\pi$
	\item[$a_i \texttt{ releases } f \texttt{ after } d \texttt{ if } \pi$] 
	akcja $a_i$ powoduje uwolnienie fluentu $f$ po $d$ chwilach od zakończenia akcji $a_i$,
	jeśli zachodzi warunek $\pi$	
	\item[$\texttt{typically } a_i \texttt{ releases } f \texttt{ after } d \texttt{ if } \pi$] 
	zazwyczaj akcja $a_i$ powoduje uwolnienie fluentu $f$ po $d$ chwilach od zakończenia akcji $a_i$,
	jeśli zachodzi warunek $\pi$	
	\item[$\pi \texttt{ triggers } a_i$] 
	jeśli zachodzi warunek $\pi$	to wykonywana jest akcja $a_i$
	\item[$\texttt{typically }\pi \texttt{ triggers } a_i$] 
	zazwyczaj jeśli zachodzi warunek $\pi$ to wykonywana jest akcja $a_i$
	\item[$\texttt{impossible } a_i \texttt{ at } t \texttt{ if } \pi$] 
	akcja $a_i$ jest niemożliwa  w chwili $t$, jeśli zachodzi warunek $\pi$	
	\item[$\texttt{always } \pi$] 
	każdy stan spełnia warunek $\pi$
\end{description}

\begin{example}
	Rozważmy przykład studenta Janka. Janek mieszka w akademiku.
	Przed wyjściem zazwyczaj bierze ze sobą kartę do wstępu do akademika i zamyka pokój.
	W chwili 10 cieć zamyka akademik i można wejść tylko z kartą.
	\begin{align*}
	&\texttt{initially } \neg roomClosed, \neg hostelClosed, inHostel, \neg hasCard \\
	&\texttt{typically } \neq hasCard \texttt{ triggers } TakeCard \\
	&\texttt{typically } Leave \texttt{ invokes } LookTheDoor \\
	&TakeCard \texttt{ causes } hasCard \\
	&Leave \texttt{ causes } \neg inHostel \\
	&Comeback \texttt{ causes } inHostel \texttt{ after } 10 \texttt{ if } hasCard \\
	&hostelClosed \texttt{ occures at } 10
	\end{align*}
\end{example}


\subsection{Scenariusze działań}
\begin{definition}
Scenariusze działań definujemy jako zbiór obserwacji i akcji:
\begin{equation}
	Sc = (OBS, ACS)\\
\end{equation}
\begin{description}
	\item[$OBS = \lbrace(\gamma_1, t_1), \dots, (\gamma_m, t_m)\rbrace$]
	obserwacja, każda obserwacja jest stanme spełniającym warunek $\gamma$ w pewnej chwili $t$
	\item[$ACS = \lbrace(a_1, t_1), \dots, (a_n, t_n) \rbrace$]
	gdzie $a_i$ -- akcje, $t_i$ -- punkt w czasie rozpoczęcia akcji
\end{description}
\end{definition}

\subsection{Semantyka}
\todo{Przpisać}

\section{Język zapytań}
Zdefiniowany język akcji może być odpytywany przez poniższy język zapytan,
który zwraca odpowiedzi \texttt{TRUE/FALSE} na następujące pytania
\begin{description}[style=nextline]
	\item[Czy w chwili $t$ realizacji scenariusza $Sc$ warunek $\gamma$ zachodzi zawsze/kiedykolwiek/na ogół?]
	$\texttt{always/ever/generally } \gamma \texttt{ at } t \texttt{ when } Sc$
	\item[Czy w chwili $t$ realizacji scenariusza $Sc$ akcja $a$ wykonywana jest zawsze/kiedykolwiek?]
	$\texttt{always/ever/generally performed } a \texttt{ at } t \texttt{ when } Sc$
	\item[Czy realizacji scenariusza $Sc$ zaangażowana jest grupa agnetów $Ag$ zawsze/kiedykolwiek?]
	$\texttt{always/ever involed } Ag \texttt{ when } Sc$	
\end{description}

\subsection{Semantyka}
\todo{Przepisać}

\section{Przykłady}
\todo{Dodać kilka przykładów z scenariuszem (+ ilustracja), kwerendą i wyjaśnieniem dlaczego tak}

\end{document}
